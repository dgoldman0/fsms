\documentclass{article}
\usepackage[utf8]{inputenc}
\usepackage{biblatex}
\bibliography{main.bib}

\title{On Fluctuation in Sexual Modes}
\author{Daniel Goldman}
\date{February 2020}

\begin{document}

\maketitle
\section{Abstract}
This call to research paper addresses some questions about intersex and differences in sexual development and issues with the binary sex modal, before constructing the idea of fluctuations in sexual modes and suggesting a few possible areas of research to better understand evolutionary dynamics of certain intersex states. While there are many conditions that fall under the category of intersex, this paper focuses on the phenomenon of SRY gene translocations, including the possibility of seemingly endosex males whose SRY gene appears on the X chromosome rather than the Y chromosome. 
\section{Introduction}
Early biological models of sex were often very simplistic. There were males, and there were females. Everyone fell into one of those two categories. Over time, biological theory evolved. The emergence of genetics as a field and the decline in cost of genetic testing allowed for the analysis of a broader proportion of the population. It became clear that not only were there genetic factors which helped to determine sex, but there were multiple configurations of these genetic factors. Most people did fall into the XX or XY groups and displayed typical sexual phenotypes. But it was also found that many did not. 

In many cases, those who did not fit into these categories were considered to have disorders, and intersex people are still often considered to have disorders. And it is true that for many, differences in sexual development can lead to health problems and a reduction of quality of life. Therefore such people are often labeled as having disorders of sexual development. However, given the offensive nature of such terminology, some have come to redefine the acronym as "differences in sexual development. These differences emerge from time to time, due to various mutations in sex determining chromosomes or due to embryological development. Cases where these changes have the potential to be passed on through multiple generations result in fluctuations in sexual modes (FSM). These fluctuations should be seen, at least in some cases, as a natural process of evolution, rather than a disease.  
\section{Binary and Bimodal Models}
The binary model of sex is often employed to discuss the topic of sexual modalities in humans. This model is simple, and easy to understand, as well as to use, and so it's not unreasonable to use it as an approximation. But the difficulty in assigning sex, based on one or a few often contradictory parameters suggests that a binary model is insufficient\cite{Legato2018}. Instead, a collection of parameters exists, and this collection is at most bimodal.

These parameters include genotype, presence or absence of a specific gametocyte and gamete, types of gonadal tissue found and in what stage of development, and related sexual characteristics. These parameters tend to cluster tightly into two groups, but there are still many outliers, and therefore the data precludes a binary model. While there are two modes, and we can define one mode as male, and the other as female, sex designation is a distribution and many people do not fall into either mode.
\subsection{Spectrum or Cloud}
Sex can be described as a spectrum, and it is not entirely unreasonable to do so. A person's biology can be described as more male-like or more female-like. However, not everyone is comfortable with the spectrum description, and there may be a slightly better option. As already mentioned, the two modes of sexual configurations involve a lot of parameters. There is no single axis that necessarily makes sense. Instead, individual data points exist within a cloud with two primary clusters indicating male and female modes. So rather than calling sex a spectrum, it can simple be called a distribution or a cloud. 
\section{Evolutionary Barriers}
Our genes are in a constant state of flux. Our sex determining genes are no different. And yet, while sex is a distribution, it does appear to be bimodal, at least in humans. Aside from the efficiency of such configurations, we can consider a few other barriers against multimodal sex modalities. First, and perhaps most importantly, there is the physical ability to copulate. That is to say that the gametes of two individuals need to actually be able to reach each other in order for the gametes to fuse. Second, the gametes themselves need to be able to fuse and produce viable offspring. If there is no viable fusion or no ability for gametes to reach one another, no reproduction occurs. And when they can reach each other and fuse, but only within two sets of sub-populations, then the populations themselves begin to diverge.

Additionally, for these modes to be stable, they would have to be able to produce a significant number of offspring with the same modes. Many intersex conditions result in partial and total infertility, but even those who do not do not generally produce intersex offspring that have fertile offspring with the same condition. 
\section{Intersex Conditions}
There are a number of intersex conditions that may be important for the understanding of FSMs. While a large number of intersex conditions result in total infertility, many others result in only partial infertility. Moreover, as technology continues to evolve, fertility issues are becoming less and less of a problem. There is even some work being done to create gametes through artificial means, so that those who do not produce any viable gamete will still be able to reproduce. 
\subsection{SRY Positive X Chromosome Mutation}
SRY translocation is a rare condition where the SRY gene, found on the Y chromosome, translocates to the X chromosome. In this discussion, an X chromosome with the SRY gene addition will be labeled M, because in those individuals with XM genotypes, it generates male-like phenotypes. The general classification of such individuals is "XX male." 

SRY positive XX individuals do not develop completely as males, because only the SRY factor is present, rather than the entire Y chromosome. However, genetically such a person is compatible with other individuals. If a stem cell could be coaxed into differentiating into a gamete, the gamete should be able to fuse with the gamete of another individual. However, the Mendelian genetics of such fusion would not be identical to that of a "typical" or "endosex" male, or an endosex female. 

Since we are talking about reproductive outcomes, barring new mutations and non sexual recombinations, let’s consider the reproductive outcomes in the case of reproduction with an endosex male. Say that the person has one “normal” X and one M. Then the offspring could be XX, XY, MX, and MY. It’s not unreasonable to say then, that an XM individual is not the same sex as an endosex XX, because the \textbf{potential} sexual outcomes are different, barring new mutations. 

However, an even better cases study may exist, though the production of such a karyotype would likely be a very rare pheonomenon in the first place, and that's the MY case. While MX would require artificial means of creating gametes and fusing them with another gamete, the situation may not be true with MY. In this case, there is an extra SRY gene, but the entire Y chromosome is also present. Therefore all the genes necessary to coordinate sexual development are present. 

Moreover, it is recognized that many individuals with an extra Y chromsome (47, XYY) are fertile\cite{BorjianBoroujeni2017}. Additionally, in a 46,MY individual, there is an even number of chromosomes, and so there would be no issue with meiotic division. Finally, there would be fewer duplicate genes and so it is likely that there would be less over expression. It is therefore not unreasonable to assume that such an individual would go about their life, never being aware that they are MY rather than XY.

However, once again potential reproductive outcomes are not identical to endosex XY males. An MY male reproducing with an endosex XX female would produce MX and XY offspring, and no XX or MY individuals. Again barring new mutations, they would have a roughly 50\% chance of producing an endosex male, a roughly 50\% chance of producing an intersex MX male. So once again, it makes at least some sense to consider that MY is not the same sex as an endosex XY individual. 
% The problem with this notion is that it would be very unusual for an MY to form in the first place. A mosaic is possible. It may also be possible under some other rare occurrances. 

\subsection{Klinefelter Syndrome and XYY syndrome}
\subsection{Swyer Snydrome}
Swyer Syndrome, also known as \cite{Dumic2008}

\section{Identifying the Fluctuation}
% Maybe change this discussion to a more general one. Rather than focusing on the SRY factor, talk about a number of different intersex cases, especially where there have been viable offspring. 
This analysis doesn't mean that intersex itself is a third sex, or that there are more than two sexes. It is not reasonable to call intersex as a whole a third sex\cite{Carpenter2018}. However, these individuals do not fall into the modes of male or female, and can be seen as belonging to an emerging, but local cluster causing the sexual modes to fluctuate. 
\subsection{A Changing Evolutionary Pressure}
Traditionally, fertility has been a significant evolutionary advantage in human populations. For instance, with high infant mortality rates, the number of offspring that a family produces must be large enough to at least ensure survival of the lineage. With a decrease in infant mortality, the selective pressure for high fertility is therefore reduced. And there is some indication that lower infant mortality rates lead to greater birth intervals and lower fertility rates\cite{vanSoest2018}. Additionally, having family members that do not produce offspring of their own can give an evolutionary advantage to the family as a whole. This idea is represented in the kin selection theory that potentially explains the persistence of genetic tendencies towards male androphilia. 

And so, as the selective pressure for greater fertility declines, so too should the selective pressure against intersex biology that results in lower fertility rates. Therefore we should expect intersex rates to increase, or at least be allowed to increase, as the need for high fertility decreases, and the evolutionary advantage for lower fertility increases.

Additionally, there is some indication that child mortality rates and fertility rates are negatively associated due to "offspring competition for parental investment" and that better socioeconomic conditions magnify this trade-off\cite{Lawson2012}. Therefore we should expect higher rates of DSD in regions of high socioeconomic conditions, and an increase in DSDs associated with improving socioeconomic conditions. 
\section{Othering and Binning}
None of this discussion is meant to suggest that intersex is leading towards an extreme divergence in the population in its current state. Human populations are still normally distributed across parameters, as a whole. But the FSM process is related to the more general evolutionary process and even more so to the process of population divergence, and population divergence tends to disallow for multimodal sex configurations. 

Concerns about "othering" intersex people should not be ignored, but it is important to recognize FSMs, in part because treating everyone as male or female can be dangerous, especially from a medical perspective. The idea that everyone should be normalized is equally dangerous as othering. For instance, consider an intersex person with XY genotype, but whose phenotype more closely mirrors that of an endosex female. Suppose they're classified as a female, for medical purposes. Then diseases that are far more likely in endosex males than endosex females may not be considered, or there may be a delay in diagnosis.

Consider Duchenne muscular dystrophy (DMD), a disease linked to the X chromosome, and which effects endosex females at a rate that is orders of magnitude smaller than the rate in endosex males. By treating the person as a female, for medical purposes, a diagnosis of DMD may be delayed by some time, because of the low probaiblity of an endosex female having the condition. Instead, for the purposes of diagnosis of sex-linked genetic conditions, the person should be treated more similarly to an endosex male.
\section{Conclusion}
Fluctuations in sexual modes is a poorly understood topic. Like the rest of our genome, our sex based genes are constantly evolving. While some forms of intersex can be seen as a disorder, many other forms may simply be part of the natural process of evolution. This process appears to be a very slow one, with no known change in sexual modes being present within the population. But we may be underestimating these fluctuations, because some forms may not be an apparent hindrance, or be obvious enough to warrant genetic testing. However, by expanding surveillance, focusing our attention on the family members of those with identified intersex configurations, we may be able to better identify these fluctuations and understand our continued evolution. 

Finally, intersex people cannot always easily be classified as either male or female, and should not be. Intersex itself should also not be seen as a third sex, even if specific intersex configurations may potentially be seen as an emerging third sex. Instead, for now there are two sexes, if sex is defined as the two modes of the sex characteristic distribution, where it is recognized that not everyone fits into one of these two modes, and that the modes themselves are under a constant state of fluctuation, and that there are individuals who do not fit into either mode, and are not either sex, but instead exist within the larger cloud or distribution.
\printbibliography
\end{document}
